\documentclass[submit]{../harvardml}

\course{CS1810-S25}
\assignment{Homework \#4}
\duedate{April 4, 2025 at 11:59 PM}

\usepackage[OT1]{fontenc}
\usepackage[colorlinks,citecolor=blue,urlcolor=blue]{hyperref}
\usepackage[pdftex]{graphicx}
\usepackage{graphicx}
\usepackage{caption}
\usepackage{fullpage}
\usepackage{soul}
\usepackage{amsmath}
\usepackage{amssymb}
\usepackage{framed}
\usepackage{color}
\usepackage{todonotes}
\usepackage{listings}
\usepackage{../common}
\usepackage{xcolor}

%%%%%%%%%%%%%%%%%%%%%%%%%%%%%%%%%%%%%%%%%%%
%% Solution environment
\usepackage{xcolor}
\newenvironment{solution}{
    \vspace{2mm}
    \color{blue}\noindent\textbf{Solution}:
}{}
%%%%%%%%%%%%%%%%%%%%%%%%%%%%%%%%%%%%%%%%%%%

\usepackage[mmddyyyy,hhmmss]{datetime}

\definecolor{verbgray}{gray}{0.9}

\lstnewenvironment{csv}{
  \lstset{backgroundcolor=\color{verbgray},
  frame=single,
  framerule=0pt,
  basicstyle=\ttfamily,
  columns=fullflexible}}{}
 
\begin{document}

\begin{center}
{\Large SVM, Clustering, and In-Depth Ethics}\\
\end{center}

\subsection*{Introduction}

This homework assignment will have you work with SVMs, clustering, and engage with the ethics lecture.

\subsection*{Resources and Submission Instructions}
We encourage you to
read Chapter 5 in the textbook to learn more about SVMs, 6.2 to review k-means clustering, and 6.3 to review HAC. % FDV (done): Fill from textbook \

Please submit the \textbf{writeup PDF to the Gradescope assignment `HW4'}. Remember to assign pages for each question.

Please submit your \textbf{\LaTeX\ file and code files to the Gradescope assignment `HW4 - Supplemental'}. 

You can use a \textbf{maximum of 2 late days} on this assignment.  Late days will be counted based on the latest of your submissions. 

\newpage

%%%%%%%%%%%%%%%%%%%%%%%%%%%%%%%%%%%%%%%%%%%%%
% Problem 1
%%%%%%%%%%%%%%%%%%%%%%%%%%%%%%%%%%%%%%%%%%%%%
\begin{problem}[Fitting an SVM by hand, 10pts]
  For this problem you will solve an SVM by hand, relying on principled rules and SVM properties. 
  For making plots, however, you are allowed to use a computer or other graphical tools.

  Consider a dataset with the following 7 data points each with $x \in \reals$ and $y \in \{ -1, +1 \}$ : \[\{(x_i, y_i)\}_{i = 1}^7 =\{(-3 , +1) , (-2 , +1 ) , (-1,  -1 ), (0, +1), ( 1 , -1 ), ( 2 , +1 ) , (3 , +1 )\}\] Consider
  mapping these points to $2$ dimensions using the feature vector $\bphi(x) =  (x, -\frac{8}{3}x^2 + \frac{2}{3}x^4)$. The hard margin classifier training problem is:
  \begin{align*}
    &\min_{\mathbf{w}, w_0} \frac{1}{2}\|\mathbf{w}\|_2^2 \label{eq:dcp} \\
    \quad \text{s.t.} \quad & y_i(\mathbf{w}^\top \bphi(x_i) + w_0) \geq 1,~\forall i \in \{1,\ldots, n\}\notag
  \end{align*}

  Make sure to follow the logical structure of
  the questions below when composing your answers, and to justify each step.

  \begin{enumerate}
    \item Plot the transformed training data in $\reals^2$ and draw the optimal decision boundary
    of the max margin classifier. You can determine this by inspection (i.e. by hand, without actually doing any calculations).

    \item What is the value of the margin achieved by the optimal
    decision boundary found in Part 1? 

    \item Identify a unit vector that is orthogonal to the decision boundary.

    \item Considering the discriminant $h(\bphi(x);\boldw,w_0)=\boldw^\top\bphi(x) +w_0$, 
    give an expression for {\em all possible} $(\boldw,w_0)$ that define
    the decision boundary. Justify your answer.

    \item Consider now the training problem for this dataset. Using your answers so far,
    what particular solution to $\boldw$ will be optimal for the
    optimization problem?

    \item What is the corresponding optimal value of $w_0$ for the $\boldw$ found in Part 5 (use your result from Part 4 as guidance)? Substitute in these optimal values and write out the discriminant function
    $h(\bphi(x);\boldw,w_0)$ in terms of the variable $x$.

    \item Which points could possibly be support vectors of the classifier?  Confirm that
    your solution in Part 6 makes the constraints above tight---that is,
    met with equality---for these candidate points.

    \item Suppose that we had decided to use a different feature mapping
    $\bphi'(x) = (x, -\frac{31}{12}x^2 + \frac{7}{12}x^4 )$.  Does
    this feature mapping still admit a separable solution?  How does
    its margin compare to the margin in the previous parts?  Based on
    this, which set of features might you prefer and why?   
  \end{enumerate}
\end{problem}

\begin{solution}
	Your solution here.
\end{solution}


\newpage

%%%%%%%%%%%%%%%%%%%%%%%%%%%%%%%%%%%%%%%%%%%%%
% Problem 2
%%%%%%%%%%%%%%%%%%%%%%%%%%%%%%%%%%%%%%%%%%%

\begin{problem}[K-Means and HAC, 20pts]

  For this problem you will implement K-Means and HAC from scratch to cluster image data. You may use \texttt{numpy} but no third-party ML implementations (eg. \texttt{scikit-learn}).

  Your job is to implement K-means and HAC on MNIST, a collection of handwritten digits, and to test whether these relatively simple algorithms can cluster similar-looking images together.

  The code in \texttt{homework4.ipynb} loads the images into your environment into two arrays -- \texttt{large\_dataset}, a 5000x784 array, will be used for K-means, while \texttt{small\_dataset}, a 300x784 array, will be used for HAC. In your code, you should use the $\ell_2$ norm (i.e. Euclidean distance) as your distance metric.

  \textbf{Important:} Remember to include all of your plots in your PDF submission!

  \begin{enumerate}
    \item Starting at a random initialization and $K = 10$, plot the
      K-means objective function (the residual sum of squares) as a
      function of iterations and verify that it never increases.

    \item For $K=10$ and for 3 random restarts, print the mean image (aka
      the centroid) for each cluster. There should be 30 total images.

    \item Repeat Part 2, but before running K-means, standardize or center
      the data such that each pixel has mean 0 and variance 1 (for any
      pixels with zero variance, simply divide by 1). For $K=10$ and 3
      random restarts, show the mean image (centroid) for each
      cluster. Again, present the 30 total images in a single
      plot. Compare to Part 2: How do the centroids visually differ? Why?

    \item Implement HAC for min, max, and centroid-based linkages. Fit
      these models to the \texttt{small\_dataset}.  For each of these 3
      linkage criteria, find the mean image for each cluster when using
      $10$ clusters. Display these images (30 total) on a single plot.

      How do the ``crispness'' of the cluster means and the digits
      represented compare to mean images for k-means?  
      Why do we only ask you to run HAC once?  

      \textbf{Important Note:} For this part ONLY, you may use
      \texttt{scipy}'s \texttt{cdist} function to calculate Euclidean
      distances between every pair of points in two arrays.

    \item For each of the HAC linkages, as well as one of the runs of your
      k-means, make a plot of ``Number of images in cluster" (y-axis)
      v. ``Cluster index" (x-axis) reflecting the assignments during the
      phase of the algorithm when there were $K=10$ clusters.

      Intuitively, what do these plots tell you about the difference
      between the clusters produced by the max and min linkage criteria?

      Going back to the previous part: How does this help explain the
      crispness and blurriness of some of the clusters?  
  \end{enumerate}
\end{problem}
\newpage
\begin{framed}
  \noindent\textbf{Problem 2} (cont.)\\
  \begin{enumerate}
  \setcounter{enumi}{5}
    \item For your K-means with $K = 10$ model and HAC min/max/centroid
      models using $10$ clusters on the \texttt{small\_dataset} images,
      use the \texttt{seaborn} module's \texttt{heatmap} function to plot
      a confusion matrix between each pair of clustering methods.  This
      will produce 6 matrices, one per pair of methods. The cell at the
      $i$th row, $j$th column of your confusion matrix is the number of
      times that an image with the cluster label $j$ of one method has
      cluster $i$ in the second method.  Which HAC is closest to k-means?
      Why might that be?

    \item Let's return to the postal service example from HW3.  Do you
      think that clustering is a good way to identify digits, that is,
      first cluster the data, and then, for any new data point, classify
      it based on its cluster?
      \begin{enumerate}
        \item In particular, do you expect the clusters to correspond with the
          true labels?  Is that a good way to evaluate clustering?
        
        \item In previous homeworks, you considered the possibility of adversarial attacks. 
        In a similar vein of thought, discuss some types of ``bad'' data that clustering algorithms 
        in particular may struggle to handle. For example, what might happen if you run clustering algorithms
        on a dataset with a few noisy outliers that are far from the rest of the dataset?
      \end{enumerate}
  \end{enumerate}
\end{framed}

\begin{solution}
	Your solution here.
\end{solution}


\newpage
%%%%%%%%%%%%%%%%%%%%%%%%%%%%%%%%%%%%%%%%%%%%%
% Problem 3
%%%%%%%%%%%%%%%%%%%%%%%%%%%%%%%%%%%%%%%%%%%%%

\begin{problem}[Ethics Assignment, 5pts]

In the previous problem, you applied K-means and HAC to the MNIST dataset of hand-written digits. In this case, choosing $K=10$ made intuitive sense because we know there are exactly ten digits, and the effectiveness of the clustering algorithm could be easily evaluated. However, clustering is widely used in more complex domains where choosing $K$ is not as straightforward.
\begin{enumerate}
  \item Consider a genetic researcher using clustering algorithms to analyze human genetic sequences. Compare this scenario to the MNIST case. Identify key differences that make selecting the hyperparameter $K$ and evaluating the performance of the algorithms more challenging in genetic research.
  \footnote{This is not a purely fictional case! In 2002, Rosenberg et al. used $K = 6$ in their study of genotypes and human population structure.}
  
  \item Some have argued that because clustering algorithms can group human genetic data into population clusters that roughly align with geographic regions, this supports the idea that racial classifications have a biological basis. Why might this conclusion be premature? Explain why the presence of genetic population clusters alone is not sufficient to establish that humans are divided into distinct biologically significant subgroups (subspecies). You should consider how selection of the hyperparameter $K$ and other model architecture decisions can bias the interpretation of genetic clustering.
  
  \item Finally, discuss one potential harm that could result from drawing such a conclusion too hastily.
\end{enumerate}
\end{problem}

\begin{solution}
	Your solution here.
\end{solution}

\newpage
%%%%%%%%%%%%%%%%%%%%%%%%%%%%%%%%%%%%%%%%%%%%%
% Name and Calibration
%%%%%%%%%%%%%%%%%%%%%%%%%%%%%%%%%%%%%%%%%%%%%

\textbf{Name}:

\textbf{Collaborators and Resources}: 

\end{document}
